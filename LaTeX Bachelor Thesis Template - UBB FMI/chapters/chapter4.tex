\chapter{Capitolul 4: Tehnologiile de pe Front-end}
\label{chap:ch3}

\section{HTML}
\label{sec:ch4sec2}
\par HTML, sau HyperText Markup Language, este un limbaj de marcare standard pentru texte care sunt destinate a fi vizualizate într-un browser web. Foile de stil în cascadă (Cascading Style Sheets - CSS) și limbajele de programare, cum ar fi JavaScript, pot ajuta în acest sens.
\par Browserele web acceptă documente HTML de pe un server web sau fișiere stocate local și le convertesc în pagini web multimedia. Inițial, HTML a furnizat indicii pentru aspectul documentului și a descris în mod logic structura unei pagini web.
\par Etichete precum <image /> și <input /> inserează imediat materiale în pagină. Alte etichete, cum ar fi <p>, înconjoară și oferă informații despre conținutul documentului și pot cuprinde subelemente, cum ar fi alte etichete. Etichetele HTML nu sunt afișate de browsere, dar sunt folosite pentru a înțelege conținutul paginii.
\par HTML permite limbajelor de scripting, cum ar fi JavaScript, să încorporeze programe care modifică comportamentul și conținutul paginilor online. CSS determină aspectul și aspectul materialului. Din 1997, World Wide Web Consortium (W3C), care gestiona standardele HTML și care acum gestionează standardele CSS, a recomandat utilizarea CSS în locul HTML-ului de prezentare explicit.

\section{CSS}
\par CSS (Cascading Style Sheets) este un limbaj de foi de stil pentru a specifica aspectul unui document scris într-un limbaj de marcare precum HTML. Alături de HTML și JavaScript, CSS este o componentă cheie a paginilor web. 
\par CSS permite separarea prezentării și a conținutului, inclusiv a aspectului, culorilor și fonturilor. Această separare poate îmbunătăți accesibilitatea conținutului și poate oferi mai multă libertate și control în definirea caracteristicilor de prezentare.Mai multe pagini web pot partaja formatarea prin definirea CSS corespunzătoare într-un fișier .css separat, ceea ce reduce la minimum complexitatea și duplicarea conținutului structural, permițând în același timp ca fișierul .css să fie memorat în memoria cache pentru a îmbunătăți performanța de încărcare a paginilor pe site-urile care partajează fișierul și formatarea acestuia.Posibilitatea de a afișa aceeași pagină de marcaje în mai multe stiluri pentru tehnici de redare distincte, cum ar fi pe ecran, pe hârtie, prin voce (prin intermediul unui browser bazat pe vorbire sau al unui cititor de ecran) și pe dispozitive tactile bazate pe Braille, este, de asemenea, posibilă prin separarea formatării și a conținutului. În cazul în care materialul este accesibil pe un dispozitiv mobil, CSS oferă reguli de formatare alternativă.

\par Denumirea de cascadă provine de la schema de prioritate specificată pentru a determina ce regulă de stil se aplică dacă mai multe reguli se potrivesc unui anumit element. Această schemă de prioritate în cascadă este previzibilă. Consorțiul World Wide Web menține standardele CSS (W3C). RFC 2318 specifică tipul de media Internet text/css (tip MIME) pentru utilizare cu CSS (martie 1998). Pentru documentele CSS, W3C oferă un serviciu gratuit de validare CSS. În afară de HTML, alte limbaje de marcare acceptă utilizarea CSS, inclusiv XHTML, XML simplu, SVG și XUL.


\section{React}
\label{sec:ch4sec1}

\par  Considerat un JavaScript framework ca si Angular sau Vue.js,de fapt React este o biblioteca source. Cel mai mult este folosit pentru proiecte mari deoarece modul de lucru cu componente ajuta la intretinerea mai usoara a aplicatiei, dar de multe ori este foosit is la aplicatii cu osingura pagina. Acesta a fost implementat imediat în fluxul de știri Facebook în 2011, după ce a fost creat de Jordan Walke, un dezvoltator de software de la Facebook. Povestea continuă un an mai târziu, când Instagram, deținută de Facebook, a introdus React. Această colecție alimentează în prezent sute de mii (dacă nu chiar milioane) de site-uri web, iar multe altele sunt lansate în fiecare zi. Această colecție alimentează în prezent sute de mii (dacă nu chiar milioane) de site-uri web, multe altele fiind lansate în fiecare zi. De fapt, odată cu lansarea React, adoptarea unor biblioteci JavaScript mici, dar puternice, a crescut vertiginos. Utilizatorii cer din ce în ce mai mult site-uri web mai rapide și mai dinamice, în timp ce dezvoltatorii optează pentru configurații contemporane, flexibile, care nu includ un număr mare de linii de cod. Pentru multe persoane, ReactJS este alegerea evidentă. Să trecem în revistă motivele cheie pentru care folosim React pentru a explica de ce. React este o bibliotecă open source, în ciuda faptului că este clasificat ca un cadru JavaScript precum Angular sau Vue.js. Este utilizat în principal pentru aplicații cu o singură pagină și interfețe web mari și sofisticate.

\par Simplitatea lui ReactJS este poate primul motiv pentru care atât de mulți oameni îl folosesc în proiectele lor. Deoarece React este o bibliotecă JavaScript, dezvoltatorilor care sunt deja familiarizați cu funcțiile JS le va fi mai ușor să înceapă cu ReactJS. Dezvoltatorii folosesc această bibliotecă pentru a crea interfețe folosind o sintaxă asemănătoare cu cea din HTML, numită JSX. Codul HTML și CSS este generat ca o consecință. API-ul React este simplu, dar puternic, iar învățarea câtorva metode fundamentale este tot ce aveți nevoie pentru a începe. Atunci când doriți să utilizați React alături de cadre JS suplimentare, cum ar fi Redux, Material UI sau Enzyme, există o mică curbă de învățare. Deși astfel de biblioteci nu fac parte din stiva React, ele oferă funcționalități suplimentare și facilitează manipularea componentelor React. Majoritatea bibliotecilor populare sunt complet documentate și nu ar trebui să ofere dificultăți niciunui dezvoltator.

\par Pentru a dezvolta interfețe utilizator, React JS utilizează tehnologiile JSX și Virtual DOM. Dezvoltatorii pot realiza aplicații pentru majoritatea platformelor și, deoarece pot vedea imediat efectele codului lor, pot vedea și înțelege mai bine ceea ce fac. React JS utilizează un limbaj cunoscut sub numele de JSX. Acesta este un plugin React care vă permite să amestecați HTML alături de JavaScript, oferind codului dvs. o mai mare flexibilitate. Majoritatea browserelor actuale sunt compatibile cu React, permițând dezvoltatorilor să își actualizeze și să își testeze DOM-ul pe mai multe platforme. Document Object Model (DOM) este o interfață de program de aplicație care este prescurtarea pentru DOM virtual (API).Acesta permite programelor să citească conținutul oricărui site web și să îl modifice în funcție de gusturile și cerințele programatorului. Orice site web care nu utilizează React JS își modifică și schimbă DOM-ul folosind HTML.
React JS poate avea un DOM virtual, care este o replică a DOM-ului original utilizat de aplicație sau de site-ul web, datorită implementării JSX. Diferența majoră dintre DOM și DOM virtual este că primul dezvăluie modificările doar după ce pagina a fost reîmprospătată, în timp ce al doilea face acest lucru în timp real, fără a necesita o reîncărcare.

\section{TypeScript}
\label{sec:ch4sec2}

\par TypeScript a fost făcut public pentru prima dată în octombrie 2012 (la versiunea 0.8), după doi ani de dezvoltare internă la Microsoft. La scurt timp după anunț, Miguel de Icaza a lăudat limba însăși, dar a criticat lipsa de sprijin IDE matur în afară de Microsoft Visual Studio, care nu era disponibil pe Linux și OS X în acel moment. Astăzi există sprijin în alte IDE, în special în Eclipsă, printr-un plug-in contribuit de Palantir Technologies. Diversi editori de text, inclusiv Emacs, Vim, Webstorm, Atom și a Microsoft Cod Visual Studio acceptă, de asemenea, TypeScript.TypeScript este o limbaj de programare dezvoltat și întreținut de Microsoft. Este un sintactic strict superset de JavaScript și adaugă opțional tastarea statică la limbă. TypeScript este conceput pentru dezvoltarea de aplicații mari și transcompilează la JavaScript.Deoarece TypeScript este un superset de JavaScript, programele JavaScript existente sunt, de asemenea, programe TypeScript valabile.TypeScript poate fi folosit pentru a dezvolta aplicații JavaScript pentru ambele partea clientului și partea de server executare (ca și în cazul Node.js sau Deno). Există mai multe opțiuni disponibile pentru transcompilare. Poate fi folosit fie TypeScript Checker implicit, sau Babel compilatorul poate fi invocat pentru a converti TypeScript în JavaScript.TypeScript acceptă fișiere de definiție care pot conține informații de tip ale bibliotecilor JavaScript existente, la fel C ++ fișierele antet poate descrie structura existentului fișiere obiect. Aceasta permite altor programe să utilizeze valorile definite în fișiere ca și cum ar fi entități TypeScript tipizate static. Există fișiere antet terțe pentru biblioteci populare, cum ar fi jQuery, MongoDB, și D3.js. Anteturi TypeScript pentru Node.js sunt de asemenea disponibile module de bază, care permit dezvoltarea programelor Node.js în TypeScript.