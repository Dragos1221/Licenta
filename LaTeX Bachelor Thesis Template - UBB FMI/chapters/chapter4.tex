\chapter{Capitolul 4: Tehnologiile de pe Front-end}
\label{chap:ch3}

\section{React}
\label{sec:ch4sec1}

\par Des considerat un JavaScript framework precum Angular sau Vue.js, React este de fapt o biblioteca de tip open source. Este folosit mai ales pentru interfate web mari, complexe dar si pentru aplicatii single-paged. Creat de Jordan Walke, un software engineer la Facebook, a fost implementat rapid in news feed-ul Facebook, in 2011. Un an mai tarziu, Instagram, aplicatia detinuta de Facebook, a implementat React si de aici incepe povestea. In ziua de azi, sute de mii (poate chiar milioane) de website-uri sunt sutinute de aceasta librarie si alte mii se nasc in fiecare zi. De fapt, de la lansarea React, am observat o crestere exploziva in utilizarea de librarii JavaScript mici ca dimensiuni, dar puternice. Utilizatorii vor sa utilizeze din ce in ce mai mult pagini web mai rapide si mai dinamice, in timp ce dezvoltatorii opteaza pentru medii moderne si flexibile fara tone de linii de cod in pachet. De aceea ReactJS este o alegere evidenta pentru multi. Ca sa explicam de ce, haideti sa recapitulam principalele motive pentru care folosim React. Desi considerat un JavaScript framework precum Angular sau Vue.js, React este de fapt o biblioteca de tip open source. Este folosit mai ales pentru interfate web mari, complexe dar si pentru aplicatii single-paged.
\par Primul lucru care face atat de multi oameni sa foloseasca ReactJS in proiectele lor este probabil simplitatea sa. React este o biblioteca JavaScript, astfel incat daca un dezvoltator este familiarizat cu functiile JS, va avea un start mai usor cu ReactJS. Cu această biblioteca, dezvoltatorii definesc interfete cu o sintaxa asemanatoare HTML numită JSX. Drept urmare, este produs cod HTML și CSS. API-ul React este foarte mic, dar puternic si tot ce trebuie safaci inainte de a incepe este sa inveti cateva functii de baza. Un pic din curba de învatare apare cand doriti sa utilizati React cu alte biblioteci JS, cum ar fi Redux, Material UI sau Enzyme. Desi nu fac parte din stiva React, astfel de biblioteci adauga functii suplimentare si va permit sa gestionati mai usor componentele React. Cele mai comune biblioteci sunt bine documentate si nu ar trebui sa creeze probleme niciunui dezvoltator.
\par React JS folosește câteva extensii numite JSX și Virtual DOM pentru a crea interfețe utilizator. Acestea permit dezvoltatorilor să le creeze pentru majoritatea platformelor și, datorită faptului că pot vedea rezultatele codului lor instantaneu, pot avea o vizualizare și o înțelegere mai bună a ceea ce fac.React JS folosește ceva numit JSX. Aceasta este o extensie a React care permite utilizarea HTML cu JavaScript și acest lucru face ca codul dvs. să fie mai versatil. React este compatibil cu majoritatea browserelor moderne, ceea ce îi ajută pe dezvoltatori să își schimbe și să testeze DOM pe diferite platforme.Virtual DOM: DOM este prescurtarea pentru Document Object Model, care este o interfață de program de aplicație (API). Permite programelor să citească conținutul oricărui site web, astfel încât să poată fi modificat în funcție de preferințele și nevoile programatorului. Orice site web care nu folosește React JS folosește HTML pentru a modifica și modifica DOM-ul său.
Datorită implementării JSX, React JS poate avea un DOM virtual, care este o copie a DOM-ului original utilizat de aplicație sau site-ul web. Principala diferență dintre DOM și DOM-ul virtual este că primul arată doar modificările după ce pagina este încărcată din nou, în timp ce acesta din urmă le arată în timp real, fără a fi nevoie să reîncărcați.

\section{TypeScript}
\label{sec:ch4sec2}

\par TypeScript a fost făcut public pentru prima dată în octombrie 2012 (la versiunea 0.8), după doi ani de dezvoltare internă la Microsoft. La scurt timp după anunț, Miguel de Icaza a lăudat limba însăși, dar a criticat lipsa de sprijin IDE matur în afară de Microsoft Visual Studio, care nu era disponibil pe Linux și OS X în acel moment. Astăzi există sprijin în alte IDE, în special în Eclipsă, printr-un plug-in contribuit de Palantir Technologies. Diversi editori de text, inclusiv Emacs, Vim, Webstorm, Atom și a Microsoft Cod Visual Studio acceptă, de asemenea, TypeScript.TypeScript este o limbaj de programare dezvoltat și întreținut de Microsoft. Este un sintactic strict superset de JavaScript și adaugă opțional tastarea statică la limbă. TypeScript este conceput pentru dezvoltarea de aplicații mari și transcompilează la JavaScript.Deoarece TypeScript este un superset de JavaScript, programele JavaScript existente sunt, de asemenea, programe TypeScript valabile.TypeScript poate fi folosit pentru a dezvolta aplicații JavaScript pentru ambele partea clientului și partea de server executare (ca și în cazul Node.js sau Deno). Există mai multe opțiuni disponibile pentru transcompilare. Poate fi folosit fie TypeScript Checker implicit, sau Babel compilatorul poate fi invocat pentru a converti TypeScript în JavaScript.TypeScript acceptă fișiere de definiție care pot conține informații de tip ale bibliotecilor JavaScript existente, la fel C ++ fișierele antet poate descrie structura existentului fișiere obiect. Aceasta permite altor programe să utilizeze valorile definite în fișiere ca și cum ar fi entități TypeScript tipizate static. Există fișiere antet terțe pentru biblioteci populare, cum ar fi jQuery, MongoDB, și D3.js. Anteturi TypeScript pentru Node.js sunt de asemenea disponibile module de bază, care permit dezvoltarea programelor Node.js în TypeScript.