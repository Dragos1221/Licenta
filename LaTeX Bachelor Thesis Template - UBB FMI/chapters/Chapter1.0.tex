\chapter{Introducere }
\section{Motivatie}
\label{sec:ch3sec1}

\par In ultimii ani internetul este intr-o continua extindere pe aproximativ toate domeniile. După cum se spune că totul are avantajele și dezavantajele sale prin urmare, odată cu extinderea domeniilor  vine supraîncărcare de informații și dificultăți în extragerea datelor.Pentru a putea trece mai usor peste aceasta problema sistemele de recomandari joaca un rol destul de important.
\par Cadrul sistemului de recomandări joacă un rol vital în navigarea pe internet de astăzi, fie că este vorba de cumpărarea unui produs de pe un site de comerț electronic sau vizionarea unui film la un serviciu video la cerere. În viața noastră de zi cu zi, depindem de recomandările oferite de alte persoane, fie din gură din gură, fie din recenziile sondajelor generale. Oamenii folosesc adesea sisteme de recomandare pe web pentru a lua decizii cu privire la elementele legate de alegerea lor. Sistemele de recomandare sunt instrumente și tehnici software al căror scop este de a face recomandări utile și sensibile unei colecții de utilizatori pentru articole sau produse care ar putea să îi intereseze. 
\par Sistemele de recomandare au devenit răspândite în ultimii ani, deoarece se ocupă de problema suprasolicitării informațiilor, sugerând utilizatorilor cele mai relevante produse dintr-o cantitate masivă de date. Pentru produsele media, recomandările de filme online colaborative încearcă să-i ajute pe utilizatori să acceseze filmele preferate prin captarea precisă a vecinilor similari dintre utilizatori sau filme din evaluările lor istorice comune. Cu toate acestea, din cauza datelor rare, selectarea vecinilor devine din ce în ce mai dificilă odată cu creșterea rapidă a filmelor și a utilizatorilor.
\par Cu alte cuvinte, sistemul de recomandare sau sistemele de recomandare aparțin unei clase de sisteme de filtrare a informațiilor care vizează prezicerea „preferinței” sau „ratingului” acordate unui articol.