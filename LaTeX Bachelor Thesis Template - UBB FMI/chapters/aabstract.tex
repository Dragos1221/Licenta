\par Due to the mass growth of information on the internet, it is quite hard to get to what we are interested in or to find something to our liking. For example in the field of cinema people often don't find a movie to their liking. The aim of this work is to create an algorithm to recommend movies to users, and to embed this algorithm in a website where the user can receive movie recommendations. The innovative part is how the difference between films with close similarities is made, and how a user can create their own list from which the algorithm is able to make film recommendations.
\par The first step in developing a paper and an application is to establish a goal and propose a solution, so the first chapter establishes the existing problem and briefly describes the chosen solution.
\par Chapter two presents an overview of the algorithm that makes movie recommendations. Here the general steps of the algorithm are explained in turn.
\par Chapter three presents all the use cases of the application: such as logging in, registering, adding a movie to the list, etc. This is quite an important chapter because it also contains a sequence diagram for each case so that the reader can understand how the application works behind the scenes. The chapter ends with a description of how the user can use the application.
\par In chapters four and five the technologies used for back-end application development are described in turn: Node Js, Javascript, MySQL, Python and Flask, as well as the front-end ones: HTML, CSS, React and Typescript.
\par Chapter six presents in detail how the application was developed, first describing how the database design was thought out, then describing the first server which is written in Node Js followed by describing the Flask server with which the movie recommendation is made, the chapter ends with the description of the front-end part.
\par The last one presents a small conclusion, but also future improvements that the application and the algorithm could have.