\chapter{Concluzii}
\section{Ce s-a obtinut}

\par Scopul aplicatiei a fost dezvoltarea unui algoritm care sa recomande filme si crearea unei aplicatii care sa ajute un utilizator sa isi poata nota toate filmele relevante pentru el, de exemplu filmele pe care el le-a vizualizat, totodata aceasta platforma sa recomande sa indrume utilizatorul spre alte filme facandu-i recomandari. Lucrarea si-a atins aceste scopuri puse la ineputul lucrari atat algoritmic cat si business. Obiectivul de business nu a fost unul principal mai mult necesar fiind pentru a putea interactiona cu algoritmul de recomandari.

\section{Ce s-a obtinut}
\par Un prim lucru ar fi experienta utilizatorului cu aplicatia dezvoltata, design-ul nu este cel mai reusit nefiin luati in calcul oameni cu dizabilitati nepunanduse accent pe utilizabilitatea interfetei.Tot odata aplicatia ar putea fi dezvoltata pentru a permite utilizatorului sa isi gestioneze mai bine filmele.
\par Pe partea de algoritm, acesta functioneaza si de multe ori face recomandari foarte utile,dar sunt momente cad da si rateuri cauza putand fiind multitudinea de cuvinte pe care filmele le au. O mai buna prelucrare a datelor ar ajuta algoritmul sa fie mult mai efficient.