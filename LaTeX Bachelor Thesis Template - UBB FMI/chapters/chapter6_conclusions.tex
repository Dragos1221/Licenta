\chapter{Concluzii}
\section{Ce s-a obtinut}

\par Scopul aplicației a fost dezvoltarea unui algoritm care să recomande filme și crearea unei aplicații care să ajute un utilizator să își poată notă toate filmele relevante pentru el, de exemplu filmele pe care el le-a vizualizat, totodată această platforma să recomande să îndrume utilizatorul spre alte filme făcându-i recomandări. Lucrarea și-a atins aceste scopuri puse la ineputul lucrări atât algoritmic cât și business. Obiectivul de business fiind unul secundar scopul acesteia fiind punerea în evidență a algoritmului de recomandări.

\section{Îmbunătățiri ale aplicatiei}
\par Un prim lucru ar fi experiență utilizatorului cu aplicația dezvoltată, design-ul nu este cel mai reușit nefiin luați în calcul oameni cu dizabilități nepunanduse accent pe utilizabilitatea interfeței. Totodată aplicația ar putea fi dezvoltată pentru a permite utilizatorului să își gestioneze mai bine filmele, sau adăugarea unor filtari pe genuri sau diferite caracteristici.
\par Pe partea algoritmului de recomandare, acesta funcționează și de multe ori face recomandări foarte utile,dar sunt momente cad da și rateuri cauza putând fiind multitudinea de cuvinte pe care filmele le au. O mai bună prelucrare a datelor ar ajută algoritmul să fie mult mai efficient. Tot la partea algoritmului de recomandare trebuie supervizat pe viitor modul în care diferențierea la similaritățile asemănătoare evoluează, poate pe viitor implicarea ratingului la calculul scorului ar trebui să fie mai mică sau de ce nu poate trebuie chiar mărită.