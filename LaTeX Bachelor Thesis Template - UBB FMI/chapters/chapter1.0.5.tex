\chapter{Cadru de lucru }
\section{Problema propusa}
\label{sec:ch3sec1}

\par Utilizatorul doreste ca pentru filmele pe care el le-a vizionat sau pe care le considera importante el sa poata primi recomandari. Algoritum calculeaza o matrice de scoruri in funtie de similaritatea cosinus si de rating ul filmelor si va returna utilizatorului filmele cu  cele mai bune scoruri pentru un anumit film sau pentru lista de filme.

\section{Solutie}
\par Solutia e compusa din mai multe parti. In primul rand, e nevoie de o baza de date populate cu filme care sa fie bine descrise astfel algorituml sa poata recomanda cele mai bune filme. Dupa o vreme dupa ce aplicatia va fi folosita aceasta va da un randament si mai bun din cauza ca utilizatorii vor intoduce rating uri. Urmatorul pas si poate unul din cele mai dificile este dezvoltarea algoritmului care va face recomandarile de filme. Ultima parte va consta din crearea unei interfete , un site web deoarece poate fi accesat si de pe telefon dar si de pe un calculator, in care utilizatorul sa isi poata intretine propia lui lista de flme. Pe baza acestor filme se vor face recomandari cu ajutorul algoritumului.