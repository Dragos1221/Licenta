\chapter{Cadru de lucru }
\section{Problema propusă}
\label{sec:ch3sec1}

\par Se dorește crearea unui algoritm destul de eficient care să poată recomandă o lista de filme pentru unul sau mai multe filme. Algoritmul trebuie pus la dispoziția utilizatorilor pintr-o interfață, tot managmantul aplicației fiind făcut de către un back-end. Cu aplicația respectivă utilizatorul ar trebui să poată să își caute filme, să le adauge la o lista, să poată vedea informații și cel mai important să poată primi diferite recomandări de filme.


\section{Soluție}
\par Soluția e compusă din mai multe părți. În primul rând, e nevoie de o bază de date populate cu filme care să fie bine descrise astfel algoritmul să poată recomanda cele mai bune filme. După un timp în care aplicația va fi folosită aceasta va da un randament și mai bun deoarece utilizatorii vor da note filmelor, iar algoritmul va diferenția filmele cu valorile apropiate ale similarităților. Următorul pas și poate unul din cele mai dificile este dezvoltarea algoritmului care va face recomandările de filme. Ultima parte va consta în crearea unei interfețe unde utilizatorul poate interacționa cu algoritmul, alegerea este  un site web deoarece acesta poate fi accesat și de pe telefon dar și de pe un calculator, iar  utilizatorului îi va fi simplu să își  poată întreține propria lui listă de flme. Pe baza acestor filme se vor face recomandări cu ajutorul algoritumului.