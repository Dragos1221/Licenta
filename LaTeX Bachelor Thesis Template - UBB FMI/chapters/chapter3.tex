\chapter{Capitolul 3: Tehnologiile de pe Back-end}
\label{chap:ch3}

\section{Node Js}
\label{sec:ch3sec1}

\par Node js este un ecosistem de JavaScript , open-source , cross-platform si care este bazat pe motorul V8 de la Chrome acesta executand cod JavaScript in afara unui browser web. Node.js permite programatorilor să folosească JavaScript pentru a crea instrumente in  linia de comandă și scripturi pentru servere  cu ajutorul carora se  generează conținutul dinamic pe paginiile web înainte ca pagina să fie transmisă către browserul utilizatorului.
\par JavaScript a fost creat ca un limbaj de programare care se baza pe browser si care avea un singur fir de execuție. A fi single-threaded înseamnă că, în același proces, se execută un singur set de instrucțiuni la un moment dat (în acest caz, browserul, sau doar fila curentă în browserele moderne). Datorita acestui lucru s-au ușurat lucrurile pentru dezvoltatorii care folosesc acest limbaj. JavaScript a fost inițial un limbaj util numai pentru adăugarea unor interacțiuni la paginile web, validări de formulare și așa mai departe - nimic care să nu necesite complexitatea multithreading-ului.
Cum funcționează in spate este destul de interesant, în comparație cu tehnicile tradiționale de servire web în care fiecare conexiune (cerere) generează un fir nou, preluând RAM de sistem și, în cele din urmă, maximizând cantitatea de RAM disponibilă, Node.js funcționează pe un singur fir, utilizând  apeluri non-blocante, permițându-i să suporte zeci de mii de conexiuni simultane ținute în bucla evenimentului
\par Cand vorbim de node js este foarte important de mentionat este suportul incorporat pentru gestionarea pachetelor utilizand NPM(node package manager), un instrument care vine în mod implicit la fiecare instalare Node.js. Modulele NPM sunt comparabile cu Ruby Gems în sensul că sunt o colecție de componente reutilizabile disponibile în mod public, care pot fi instalate cu ușurință dintr-un depozit online și includ gestionarea versiunilor și a dependențelor. Oricine poate contribui la ecosistemul de module prin publicarea propriului modul pentru a fi inclus în depozitul npm.

\section{Javascript}
\label{sec:ch3sec2}

\par Limbajul de programare JavaScript și-a petrecut cea mai mare parte a vieții în interiorul browserele web. A început ca un limbaj de scripting de bază pentru ajustarea celor mai fine aspecte ale paginilor web, dar acum a evoluat într-un limbaj sofisticat cu o gamă largă de aplicații și biblioteci. Mulți furnizori de browsere precum Mozilla și Google au început să pompeze resurse în timpii de rulare rapid JavaScript, iar browserele au obținut motoare JavaScript mult mai rapide ca un rezultat.În 2009, a apărut Node.js. Node.js a scos V8, puternicul motor JavaScript Google Chrome, din browser și i-a permis să ruleze pe servere.În browser, dezvoltatorii nu au avut de ales decât să aleagă JavaScript. Dezvoltatorii pot folosi acum JavaScript în loc de Ruby, Python, Java și alte limbaje pentru a crea aplicații pe partea serverului. Deși JavaScript poate să nu fie cel mai bun limbaj pentru toată lumea, Node.js oferă o mulțime de avantaje. Pentru început, motorul JavaScript V8 este rapid, iar Node.js suportă codarea asincronă, care accelerează codul evitând în același timp problemele de multithreading. Dezvoltatorii nu trebuie să facă niciun fel de schimbare de context atunci când trec de la client și Server. 

\section{MySQL}
\label{sec:ch3sec3}

\par MySQL este un sistem de gestionare a bazelor de date relaționale dezvoltat de MySQL AB din Suedia și publicat sub Licența Publică Generală GNU. Este cel mai utilizat sistem de gestionare a bazelor de date open-source în prezent și reprezintă o parte importantă a stivei LAMP (Linux, Apache, MySQL, PHP).
\par Deși este folosit foarte des împreună cu limbajelel de programare JAVA, PHP, cu MySQL pot construi aplicații în orice limbaj major. Există multe scheme API disponibile pentru MySQL ce permit scrierea aplicațiilor în numeroase limbaje de programare pentru accesarea bazelor de date MySQL. O interfață de tip ODBC denumită MyODBC permite altor limbaje de programare ce folosesc această interfaţă, să utilizeze bazele de date MySQL cum ar fi ASP sau Visual Basic  (Molnar). 
\par Multe manuale susțin că MySQL este simplu de învățat și de utilizat în comparație cu alte programe de gestionare a bazelor de date; de exemplu, comanda exit este simplă: "exit" sau "quit". 
\par Pentru a administra bazele de date MySQL se poate folosi modul linie de comanda sau se pot descărca de pe internet diferite programe ce creează o interfață grafică: MySQL Administrator şi MySQL Query Browser. Un alt instrument de management al acestor baze de date este aplicaţia SQL Manager. 
\par MySQL este un server de baze de date care este extrem de rapid, fiabil și simplu de utilizat. A fost creat pentru a gestiona baze de date uriașe mult mai rapid decât alternativele anterioare.
\par Software-ul de baze de date MySQL este un sistem client/server care include un server MySQL cu mai multe fire și o varietate de aplicații client și biblioteci, precum și instrumente de administrare și o varietate de interfețe de programare a aplicațiilor.
 
\section{Python}
\label{sec:ch3sec4}

\par Python este un limbaj de programare dinamic semantic, de nivel înalt, orientat pe obiecte și interpretat. Structurile sale de date integrate de nivel înalt, împreună cu tastarea dinamică și legarea dinamică, îl fac ideal pentru crearea rapidă de aplicații, precum și pentru utilizarea ca limbaj de scripting sau ca punct de legătură între componentele existente. Sintaxa de bază a limbajului Python, ușor de învățat, acordă prioritate lizibilității, reducând costurile de întreținere a software-ului. Modulele și pachetele sunt suportate de Python, ceea ce facilitează modularitatea programelor și reutilizarea codului. Interpretorul Python și biblioteca standard extinsă pot fi descărcate și distribuite gratuit în formă sursă sau binară pentru toate platformele majore.
\par Python este popular în rândul programatorilor datorită productivității îmbunătățite pe care o oferă. Ciclul editare-testare-depanare este foarte rapid, deoarece nu există o fază de compilare. Depanarea scripturilor Python este simplă: o problemă de segmentare nu este niciodată cauzată de o greșeală sau de o intrare incorectă. În schimb, atunci când interpretorul găsește o greșeală, aruncă o excepție. Interpretul produce o urmă de stivă dacă aplicația nu reușește să detecteze eroarea. Se pot verifica variabilele locale și globale, se pot evalua expresii arbitrare, se pot crea puncte de întrerupere, se poate parcurge codul rând pe rând și așa mai departe cu un depanator la nivel de sursă. Depanatorul este construit în Python, demonstrând capacitățile introspective ale limbajului. Pe de altă parte, adăugarea câtorva comenzi de tastare la codul sursă este frecvent cea mai rapidă metodă de depanare a unui program: ciclul rapid editare-testare-depanare face ca această tehnică de bază să fie foarte eficientă.

\section{Flask}
\label{sec:ch3sec5}

\par Flask este un framework web scris în Python care este clasificat ca un microframe, deoarece nu necesită anumite instrumente sau biblioteci.Nu are un strat de abstractizare a bazei de date, validarea formularelor sau orice alte componente în care bibliotecile terțe preexistente oferă funcții comune. Cu toate acestea, Flask acceptă extensii care pot adăuga caracteristici ale aplicației ca și cum ar fi implementate în Flask în sine. Există extensii pentru mapere relaționale obiect, validarea formularelor, gestionarea încărcărilor, diverse tehnologii de autentificare deschise și mai multe instrumente comune legate de cadru.