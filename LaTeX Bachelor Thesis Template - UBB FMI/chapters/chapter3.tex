\chapter{Capitolul 3: Tehnologiile de pe Back-end}
\label{chap:ch3}

\section{Node Js}
\label{sec:ch3sec1}

\par Node js este un ecosistem de JavaScript , open-source , cross-platform si care este bazat pe motorul V8 de la Chrome acesta executand cod JavaScript in afara unui browser web. Node.js permite dezvoltatorilor să folosească JavaScript pentru a scrie instrumente de linie de comandă și pentru scripturi de pe partea de server, rularea scripturilor pe partea de server pentru a produce conținut dinamic al paginii web înainte ca pagina să fie trimisă în browserul web al utilizatorului.
\par JavaScript a fost conceput ca un limbaj de programare cu un singur fir care rulează într-un browser. A fi un singur fir înseamnă că numai un singur set de instrucțiuni este executat în orice moment în același proces (browserul, în acest caz, sau doar fila curentă în browserele moderne).
Acest lucru a făcut lucrurile mai ușoare pentru implementare și pentru dezvoltatorii care folosesc limba. JavaScript a fost inițial un limbaj util numai pentru adăugarea unor interacțiuni la paginile web, validări de formulare și așa mai departe - nimic care să nu necesite complexitatea multithreading-ului.
Cum funcționează sub capotă este destul de interesant. În comparație cu tehnicile tradiționale de servire web în care fiecare conexiune (cerere) generează un fir nou, preluând RAM de sistem și, în cele din urmă, maximizând cantitatea de RAM disponibilă, Node.js funcționează pe un singur fir, utilizând  apeluri non-blocante, permițându-i să suporte zeci de mii de conexiuni simultane ținute în bucla evenimentului
\par Cand vorbim de node js este foarte important de mentionat este suportul incorporat pentru gestionarea pachetelor utilizand NPM(node package manager), un instrument care vine în mod implicit la fiecare instalare Node.js. Ideea modulelor NPM este destul de similară cu cea a Ruby Gems : un set de componente reutilizabile disponibile public, disponibile prin instalare ușoară printr-un depozit online, cu gestionarea versiunilor și dependenței. Ecosistemul modulului este deschis tuturor și oricine își poate publica propriul modul care va fi listat în depozitul npm.

\section{Javascript}
\label{sec:ch3sec2}

\par În cea mai mare parte a vieții sale, limbajul de programare JavaScript a trăit în interiorul web browsere. A început ca un limbaj de scriptare simplu pentru modificarea detaliilor mici ale pagini web, dar a devenit un limbaj complex, cu o mulțime de aplicații și biblioteci. Mulți furnizori de browsere precum Mozilla și Google au început să pompeze resurse în timpii de rulare rapid JavaScript, iar browserele au obținut motoare JavaScript mult mai rapide ca
un rezultat.În 2009, a apărut Node.js. Node.js a scos V8, puternicul motor JavaScript Google Chrome, din browser și i-a permis să ruleze pe servere.În browser, dezvoltatorii nu au avut de ales decât să aleagă JavaScript. Pe lângă Ruby, Python, Java, și alte limbi, dezvoltatorii ar putea alege acum JavaScript atunci când dezvoltă aplicații de pe server.Este posibil ca JavaScript să nu fie limbajul perfect pentru toată lumea, dar Node.js are adevărat beneficii. În primul rând, motorul V8 JavaScript este rapid, iar Node.js încurajează un stil de codare asincron, făcând codul mai rapid evitând în același timp coșmarurile cu mai multe fire. JavaScript avea, de asemenea, o mulțime de biblioteci utile datorită popularității sale. Cu exceptia Cel mai mare beneficiu al Node.js este capacitatea de a partaja codul între browser și server. Dezvoltatorii nu trebuie să facă niciun fel de schimbare de context atunci când trec de la client și Server. 

\section{MySQL}
\label{sec:ch3sec3}

\par MySQL este un sistem de gestiune al bazelor de date relațional, produs de compania suedeză MySQL AB și distribuit sub Licență Publică Generală GNU. Este cel mai popular SGBD open-source la ora actuală, fiind o componentă cheie a stivei LAMP(Linux, Apache, MySQL, PHP). 
\par Deși este folosit foarte des împreună cu limbajelel de programare JAVA, PHP, cu MySQL pot construi aplicații în orice limbaj major. Există multe scheme API disponibile pentru MySQL ce permit scrierea aplicațiilor în numeroase limbaje de programare pentru accesarea bazelor de date MySQL. O interfață de tip ODBC denumită MyODBC permite altor limbaje de programare ce folosesc această interfaţă, să utilizeze bazele de date MySQL cum ar fi ASP sau Visual Basic  (Molnar). 
\par În multe cărți de specialitate este precizat faptul că MySQL este destul de ușor de învățat și folosit în comparație cu multe din aplicațiile de gestiune a bazelor de date, ca exemplu comanda de ieşire fiind una simplă și evidentă: „exit” sau „quit”. 
\par Pentru a administra bazele de date MySQL se poate folosi modul linie de comanda sau se pot descărca de pe internet diferite programe ce creează o interfață grafică: MySQL Administrator şi MySQL Query Browser. Un alt instrument de management al acestor baze de date este aplicaţia SQL Manager. 
\par Serverul de baze de date MySQL este foarte rapid, fiabil și ușor de utilizat. Inițial a fost dezvoltat pentru a manipula baze de date de dimensiuni mari mult mai rapid decât soluțiile existente. 
\par MySQL Database Software este un sistem client/server ce constă într-un server MySQL multi-threaded care suportă diferite programe client și biblioteci, unelte administrative şi o gamă largă de interfețe pentru programarea aplicațiilor.
 
\section{Python}
\label{sec:ch3sec4}

\par Python este un limbaj de programare interpretat, orientat spre obiecte, la nivel înalt, cu semantică dinamică. Structurile sale de date încorporate la nivel înalt, combinate cu tastarea dinamică și legarea dinamică, îl fac foarte atractiv pentru dezvoltarea rapidă a aplicațiilor, precum și pentru utilizarea ca limbaj de scriptare sau lipici pentru a conecta componentele existente împreună. Sintaxa simplă, ușor de învățat a Python accentuează lizibilitatea și, prin urmare, reduce costul întreținerii programului. Python acceptă module și pachete, ceea ce încurajează modularitatea programului și reutilizarea codului. Interpretul Python și biblioteca standard extinsă sunt disponibile sub formă sursă sau binară fără taxe pentru toate platformele majore și pot fi distribuite în mod liber.
\par Adesea, programatorii se îndrăgostesc de Python din cauza productivității crescute pe care o oferă. Deoarece nu există nicio etapă de compilare, ciclul de editare-test-depanare este incredibil de rapid. Depanarea programelor Python este ușoară: o eroare sau o intrare greșită nu va provoca niciodată o eroare de segmentare. În schimb, atunci când interpretul descoperă o eroare, ridică o excepție. Când programul nu prinde excepția, interpretul imprimă o urmă de stivă. Un depanator la nivel de sursă permite inspectarea variabilelor locale și globale, evaluarea expresiilor arbitrare, setarea punctelor de întrerupere, trecerea prin cod o linie la un moment dat și așa mai departe. Depanatorul este scris chiar în Python, mărturisind puterea introspectivă a lui Python. Pe de altă parte, deseori cel mai rapid mod de a depana un program este să adăugați câteva instrucțiuni de tipărire la sursă: ciclul rapid de editare-test-depanare face această abordare simplă foarte eficientă.

\section{Flask}
\label{sec:ch3sec5}

\par Flask este un framework web scris în Python care este clasificat ca un microframe, deoarece nu necesită anumite instrumente sau biblioteci.Nu are un strat de abstractizare a bazei de date, validarea formularelor sau orice alte componente în care bibliotecile terțe preexistente oferă funcții comune. Cu toate acestea, Flask acceptă extensii care pot adăuga caracteristici ale aplicației ca și cum ar fi implementate în Flask în sine. Există extensii pentru mapere relaționale obiect, validarea formularelor, gestionarea încărcărilor, diverse tehnologii de autentificare deschise și mai multe instrumente comune legate de cadru.